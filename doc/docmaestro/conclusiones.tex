El aspecto más interesante de esta práctica podría ser el hecho de que había que llegar a otro nivel de abstracción para comprender todo el
funcionamiento de las aplicaciones para redes y web en este caso. Fue muy claro que las dificultades se presentan dependiendo del lenguaje en 
que se decida trabajar, puesto que el manejo de sockets y esquemas requeridos para desarrollos en telemática requieren tener mayor dominio en
lenguajes como C o C++, mientras que hay otros que proveen todas las herramientas para que el trabajo con estas nuevas funciones sea casi
transparente.
A nivel de la se necesitó investigar conceptos que no se habían visto anteriormente en la carrera, todo el manejo de procesos, hilos y segmentación en
cuanto a las peticiones fueron grandes retrasos en el desarrollo de la práctica.\\

