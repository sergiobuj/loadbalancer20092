%%introducción
Los métodos de balanceo de cargas surgen como la solución para equilibrar el trabajo realizado por cualquier unidad que forme parte de un grupo
que trata de prestar un servicio como uno solo. El balanceo de carga se usa en discos duros, procesadores, redes, donde para el caso de los
procesadores, la unidad que tiene menor número de tareas o que tiene más capacidad para asumir otra tarea obtiene el trabajo siguiente.
Los criterios para determinar quien debe recibir esa nueva tarea se determina mediante algún método preestablecido, de los cuales el más sencillo es
el método de Round Robin.
En la práctica el objetivo era implementar uno de estos balanceadores de carga para un grupo de servidores web, en donde la carga son las peticiones
a los recursos que se encuentran replicados en este grupo con la implementación de cuatro métodos diferentes para solucionar este problema. Los
métodos implementados son el Round Robin, menor número de conecciones actuales, menor carga en servidor y por último un método propio
propuesto por el grupo.
Los pasados métodos de balanceo se encuentran explicados y acompañados de un seudo código sencillo en este documento.
Más adelante se explica también la forma en que se opera el balanceador de carga implementado y bajo que condiciones se pueden obtener los
resultados deseados.
