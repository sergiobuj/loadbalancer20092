Todas las pruebas del balanceador de cargas por Round Robin fueron a pequeña escala, el ambiente de pruebas fueron dos portátiles, donde cada
portátil corría dos servidores web con páginas de inicio de desarrollos en blanco de Ruby on Rails y drupal. Una de las máquinas ejecutaba el
balanceador de cargas y desde las dos se hacían peticiones que recibía el balanceador de carga.\\

Las máquinas corrieron con estas configuraciones:
Portátil 1: IP:10.0.1.5
\begin{center}
\begin{tabular}{ l r }
Aplicación & Puerto \\ \hline
ROR &  4000 \\
Drupal  & 5000\\
LoadBalancer & 80 \\
\end{tabular}
\end{center}
Esta máquina realiza las peticiones mediante un explorador Firefox.\\ \\

Portátil 2: IP:10.0.1.4
\begin{center}
\begin{tabular}{ l r }
Aplicación & Puerto \\ \hline
ROR  & 4000 \\
Drupal  & 5000\\
\end{tabular}
\end{center}
Esta máquina realiza las peticiones mediante un explorador Safari.\\ \\



Las pruebas realizadas al método de balanceo LBS que requiere un agente en el servidor se realizó con dos máquinas, un portátil y un servidor.
En las dos máquinas se corrió una aplicación en blanco, para el servidor se corrió en Apache y en el portátil una aplicación en blanco en Ruby on Rails
con servidor Webrick.\\
La prueba consistió en correr un ``fork bomb'' con el siguiente código en C:

\begin{lstlisting}[language=C]
int main()
{
 int i=10000000;
 while(i--)
  fork();
 return 0;
}
\end{lstlisting}

El cual al cabo de unos segundos eleva la carga del procesador del servidor a un punto en el que podíamos observar cuando dejaba de enviar las
peticiones al servidor y cambiaba al portátil, también después de detener la ``bomba de procesos'' veíamos como el balanceador enviaba las nuevas
peticiones al servidor.\\

