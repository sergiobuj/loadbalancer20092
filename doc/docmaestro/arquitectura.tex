La arquitectura consiste en siempre tener un proceso que sirva de listener, y que cada vez que un cliente que se intente conectar, el listener haga fork(),
para así siempre tener el proceso en el puerto deseado, y que simplemente sea un nuevo proceso que se encargue de manejar la petición. Ahora en el
proceso hijo no hace falta listener por lo que se libera ese recurso, ahora en los procesos hijos se debe seleccionar por algún método en particular uno
de los back-end, una vez escogido uno de estos, establecemos la conexión con el back-end en el mismo proceso y en teoría solo bastaría con transferir
los buffers entre los sockets, y al final cerrar las conexiones y cerras los procesos.\\

