Los protocolos usados en este balanceador de carga son TCP para toda la conexión entre los clientes y el balanceador de carga, también es TCP entre
el load balancer y los servidores finales.\\
Se requirió la implementación de un protocolo para el manejo y envío de mensajes entre los agentes que residen en los servidores y el balanceador
 de cargas. Los mensajes que se intercambiarán determinan el tipo información que maneja el agente.\\
 
 Protocolo Balanceador de Carga-Agente
 
 \begin{itemize}
 \item Definición y especificaciones del protocolo.\\
El servicio que presta este protocolo es el de envío de mensajes entre el agente que está siendo ejecutado en un servidor y el balanceador de carga.
El fin de este servicio es que el balanceador pueda disponer de información sobre el servidor y sus procesos en las veces que sea necesario hasta
obtener un servidor que sea buen candidato para recibir la petición.\\
El balanceador de carga sabe que puede encontrar a cualquier agente disponible en el puerto nueve mil (9000) de cualquiera de los servidores
que puede usar para realizar el equilibrio de las peticiones.
Por estar soportados en TCP, este protocolo no se preocupa por pérdida de información, ordenamiento u otros.
  \item Suposiciones del entorno.\\
  Las condiciones ideales para el funcionamiento del protocolo serían las siguientes:
  \begin{enumerate}

\item Para el correcto funcionamiento del protocolo y de los agentes, las máquinas que los ejecuten deben estar corriendo sobre un sistema operativo
Linux, ya que la información que se toma de cada servidor se toma con comandos especiales de este sistema operativo.

\item Los puertos 9000 de cada máquina que hace parte del grupo de servidores que está siendo equilibrados en carga deben estar disponibles
para poder que el agente se ejecute sobre él. El agente necesita el puerto 9000 sólo para recibir las peticiones del balanceador.

\end{enumerate}
\item Vocabulario de mensajes.\\
   Los mensajes enviados desde el balanceador al agente son:``lc'' y ``lbs''. Y para cada uno hay una respuesta por parte del balanceador que consiste
   en un string de tres números, del cual sólo importa el primero, y un valor flotante indicando carga del procesador respectivamente.
   ``lc'' representa el método de balanceo por menor número de conexiones activas. y ``lbs'' es el método del servidor con la menor carga o
   ``Least Busy Server''
\item Codificación o formato de mensajes.\\
    Los mensajes se querían implementar lo más cortos posibes, por eso, para pedir una respuesta que pueda ser usada en un método de menor número
    de conexiones se debe enviar al agente la palabra ``lc'' y para que la respuesta del agente sea útil para un método de ``Least Busy Server'', la palabra
    a enviar debe ser ``lbs'', (comillas para aclarar).
\item Reglas de procedimiento.\\
El procedimiento que se debe seguir para el uso de este protocolo se basa en abrir un socket desde el balanceador de carga y enviar inmediatamente
la palabra relacionada con el método que se está usando para hacer el equilibrio. Después la respuesta que venga por ese socket es lo que se usará
para realizar el equilibrio.

\end{itemize}